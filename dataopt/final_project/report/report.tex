% !TEX TS-program = pdflatex
% !TEX encoding = UTF-8 Unicode

% This is a simple template for a LaTeX document using the "article" class.
% See "book", "report", "letter" for other types of document.

\documentclass[11pt]{article} % use larger type; default would be 10pt

\usepackage[utf8]{inputenc} % set input encoding (not needed with XeLaTeX)

%%% Examples of Article customizations
% These packages are optional, depending whether you want the features they provide.
% See the LaTeX Companion or other references for full information.

%%% PAGE DIMENSIONS
\usepackage{geometry} % to change the page dimensions
\geometry{a4paper} % or letterpaper (US) or a5paper or....
% \geometry{margin=2in} % for example, change the margins to 2 inches all round
% \geometry{landscape} % set up the page for landscape
%   read geometry.pdf for detailed page layout information

\usepackage{setspace}
\setstretch{1.5}

\usepackage{graphicx} % support the \includegraphics command and options

\usepackage{biblatex} %Imports biblatex package
\addbibresource{sources.bib} %Import the bibliography file

% \usepackage[parfill]{parskip} % Activate to begin paragraphs with an empty line rather than an indent

%%% PACKAGES
\usepackage{booktabs} % for much better looking tables
\usepackage{array} % for better arrays (eg matrices) in maths
\usepackage{paralist} % very flexible & customisable lists (eg. enumerate/itemize, etc.)
\usepackage{verbatim} % adds environment for commenting out blocks of text & for better verbatim
\usepackage{subfig} % make it possible to include more than one captioned figure/table in a single float
% These packages are all incorporated in the memoir class to one degree or another...

\usepackage{amsmath}


%%% HEADERS & FOOTERS
\usepackage{fancyhdr} % This should be set AFTER setting up the page geometry
\pagestyle{fancy} % options: empty , plain , fancy
\renewcommand{\headrulewidth}{0pt} % customise the layout...
\lhead{}\chead{}\rhead{}
\lfoot{}\cfoot{\thepage}\rfoot{}

%%% SECTION TITLE APPEARANCE
\usepackage{sectsty}
\allsectionsfont{\sffamily\mdseries\upshape} % (See the fntguide.pdf for font help)
% (This matches ConTeXt defaults)

%%% ToC (table of contents) APPEARANCE
\usepackage[nottoc,notlof,notlot]{tocbibind} % Put the bibliography in the ToC
\usepackage[titles,subfigure]{tocloft} % Alter the style of the Table of Contents
\renewcommand{\cftsecfont}{\rmfamily\mdseries\upshape}
\renewcommand{\cftsecpagefont}{\rmfamily\mdseries\upshape} % No bold!

%%% END Article customizations

%%% The "real" document content comes below...

\title{Final project report\\
	   Portfolio optimization}
\author{Juha Reinikainen}
%\date{} % Activate to display a given date or no date (if empty),
         % otherwise the current date is printed 

\begin{document}
\maketitle

\section{Problem}

Portfolio optimization involves deciding how to use the available investment budget to maximize the total value of the investment and minimize its risk \cite{van2021parden}. The investment budget is allocated to assets which can be for example stocks, gold, foreign exchange, real estate, bonds and cryptocurrencies \cite{faizan2019multiobjective}. For simplicity only stocks are considered in this project.

The problem is difficult because there is a large number of possible assets to include in the portfolio and even larger number of ways to divide the budget among them. Investing also has a lot of uncertainty as stock prices are affected by real world events which are hard to capture in the model \cite{du2020new}.

\section{Data}

Data-driven approach to this problem involves predicting expected return and risk based on historical time-series data of stock prices over given time range, possibly years. Stock price can be sampled e.g. daily, weekly, monthly or quaterly. The stocks that are included in the data-set need to be selected.

The stock price data was collected from Yahoo Finance by calling its api with different stocks and merging these into one dataset. For example the weekly prices between 2021-05-09 20:55:46 and 2022-05-09 20:55:46 can be queried using following url.

\url{https://query1.finance.yahoo.com/v7/finance/download/ADS.DE?period1=1620582946&period2=1652118946&interval=1wk&events=history&includeAdjustedClose=true}

Where perdiod1 and period2 are the start and end times of the range as unix timestamps. 

The stocks included in the problem where the stocks in eurostoxx 50 index.


\section{Modelling}

The problem can be modelled as a two objective optimization problem maximing expected return of the investment and minimizing its risk. 

\subsection{Variables}

The decision vector consist of proportions of total budget allocated to n stocks $w = (w_1, w_2, ..., w_n)$ and binary variable for each stock which indicates whether the corresponding stock is included in the portfolio $y=(y_1, y_2,..., y_n)$. 

\subsection{Constraints}

It's assumed that the whole budget is used so the sum of weights should add up to one for the weight where the corresponding boolean flag is 1.

Boundary constraint requires that the weights of each stock is between $w_{min}$ and $w_{max}$. Maximum limit makes it so that all budget is not allocated to too small number of stocks leading to diverse portfolio. Too small weights typically have little impact on the performance and weak liquidity and can be costly in respect to brokerage fess or monitoring costs \cite{ertenlice2018survey}.

Cardinality constraint requires that the number of stocks included in the portfolio is between some two numbers $C_{min}$ and $C_{max}$. 

\subsection{Objectives}

Popular way to model these objectives is Markowitz model which is also known as mean-variance model \cite{kolm201460}.

Given time series data of n stocks for time period of 0...T with stock price p(t, i) of stock i at time t. prices of stock i are a series $x_i$.

$x_i = (p(0, i), p(1, i), ..., p(T, i))$\\

Return of an investment between time t-1 to t for stock i is calculated by.

$roi(t,i) = \frac{p(t,i) - p(t-1,i)}{p(t-1,i)}$

Expected return for stock i is calculated as mean of each individual roi.

$roi(i) = \frac{roi(1, i) + roi(2, i) + ... + roi(T, i)}{T-1}$

Expected return for n stocks is calculated as weighted sum of all individual returns.

$er(w,y) = \Sigma_{i=1}^{n} (w_i * y_i * roi(i))$

Picking stocks where the price changes a lot is risky. This problem is increased by picking stocks where this happens similarly. This can be modelled using covariance of roi values \cite{kolm201460}. n x n square covariance matrix Cov contrains covariances of each roi(i) values. Given weights and inclusion flags, the risk for portfolio is given by.

$risk(w,y) = (w^T * y) Cov (w * y)$

\subsection{Problem}

Putting all this together the whole problem is.

\begin{equation}
\begin{split}
minimize \{ risk(w,y), -er(w,y) \}\\
\Sigma_{i=1}^n (w_i * y_i) = 1\\
w_{min} \leq w_i \leq w_{max}, i = 1...n\\
C_{min} \leq \Sigma_{i=1}^n y_i \leq C_{max}\\
y_i \in \{0,1\}, i = 1...n\
\end{split}
\end{equation}


\section{Algorithm and settings}

NSGA-II algorithm is used to solve the multiobjective constrained mixed-integer problem. 

For the first n real variables simulated binary crossover and polynomial mutation are used. For the last n binary variables two-point binary crossover and bitflip mutation are used. Crossover probability is set to 1, mutation probability is set to $\frac{1}{50}$. Distribution index for real mutation and crossover are set to 3. 

The constraint that the weight need to add up to one can be enforced using repair method \cite{kaucic2019portfolio}. The weights are first clamped to $w_{min}$ $w_{max}$ range. Then weight vector is element-wise multiplied by vector y to remove unselected stocks. Then each weight is divided by sum of all weights. It's easy to see that now weights add up to one. Special case where all weights are zero can be handled by assigning all weights with value $\frac{1}{n}$.

\begin{equation}
\begin{split}
&\frac{w_1}{w_1 + w_2 + ... + w_n} + \frac{w_2}{w_1 + w_2 + ... + w_n} + \frac{w_n}{w_1 + w_2 + ... + w_n}\\
&= \frac{w_1 + w_2 + ...  +w_n}{w_1 + w_2 + ... + w_n}\\
&= 1
\end{split}
\end{equation}



\section{Results}


\printbibliography %Prints bibliography

\end{document}
